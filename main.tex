\documentclass[12pt, twoside]{report}
\usepackage[nottoc,notlot,notlof]{tocbibind}
\usepackage[italian]{babel}
\usepackage[a4paper,width=150mm,top=25mm,bottom=30mm,left=25mm,right=25mm]{geometry}
\usepackage[utf8]{inputenc}
\usepackage{graphicx}
\usepackage{multicol}
\usepackage{fancyhdr}

\usepackage{blindtext}

% Percorso immagini
\graphicspath{ {images/} }

\author{Nome Cognome}
\date{gg Mese aa}

\fancypagestyle{unica}{
	\fancyhf{}
	\renewcommand{\headrulewidth}{0pt}
	% Header
	\fancyhead[LE,RO]{\nouppercase{\leftmark}}
	% Footer
	\fancyfoot[LE,RO]{\thepage}
}

\fancypagestyle{plain}{
	\fancyhead{}
	\renewcommand{\headrulewidth}{0pt}
	% Footer
	\fancyfoot[LE,RO]{\thepage}
}

\begin{document}

\pagenumbering{gobble}

\begin{titlepage}
    \begin{center}
        \vspace*{1cm}
        
        \includegraphics[width=3cm]{Logo_UNICA.png}
        
        \vspace{0.5cm}
        \Large
        \textbf{UNIVERSITA' DEGLI STUDI DI CAGLIARI}
        
        \vspace{0.5cm}
        \normalsize
        \textbf{FACOLTÀ DI SCIENZE}
        
        \vspace{0.5cm}
        Corso di Laurea in Informatica
        
        \vspace{4cm}
        
        \LARGE
        \textbf{Titolo}
        
        \vspace{2cm}
        
        
		    \begin{multicols}{2}
		    \normalsize
		    \textbf{Docente di riferimento}
		    \newline
		    Prof. Nome Cognome
		    \newline
		   

		    \textbf{Candidato:}
		    
		    Nome Cognome
		    
		    \textbf{(matr. 55555)}
		    \end{multicols}
        
        \vspace{4cm}
        
        \normalsize
        ANNO ACCADEMICO xxxx - xxxx
        
    \end{center}
\end{titlepage}


% Sommario
\tableofcontents

\pagenumbering{gobble}

\pagestyle{unica}

% Mostra solo il nome del capitolo
\renewcommand{\chaptermark}[1]{%
\markboth{#1}{}}

\chapter{Introduzione}
\pagenumbering{arabic}
\thispagestyle{unica}
\input{chapters/introduction}



\chapter{Conclusioni}
\thispagestyle{unica}
\input{chapters/chapter02}

\pagestyle{plain}
\chapter*{Ringraziamenti}
Voglio ringraziare..

\appendix
\chapter{Ambiente di lavoro}
\input{chapters/appendix}

\begin{thebibliography}{99}

\iffalse
Libri
Autore, *Titolo*, Numero di edizione (se diversa dalla prima), Luogo di pubblicazione, Editore,
Anno di pubblicazione.

Per gli autori si cita di solito prima il cognome e poi l'iniziale del nome, separati da virgola.

Quando gli autori sono due, si indicano entrambi nell'ordine in cui appaiono sullo scritto, separati da virgola; quando sono più di due si possono indicare tutti oppure solo il primo, facendo seguire la dicitura "e altri".

Per le opere straniere ricordare che le parole sostantivi, verbi, aggettivi dei titoli inglesi hanno
le iniziali maiuscole.

Articoli su periodici e riviste
Autore, "Titolo articolo", *Nome della rivista*, Volume(Numero), Anno, Pagine contenenti
l'articolo.

Capitoli di libri, atti di conferenze
Autore, "Titolo articolo", in: *Titolo dell'opera o del congresso*, NomeEditor (ed.), Eventuale volume, Luogo di pubblicazione, Editore, Anno di pubblicazione, Pagine contenenti la porzione interessata.

Citazioni di documenti tratti dal WWW (URL)
Autore, "Titolo articolo", Anno, URL, data dell’ultima consultazione.
\fi

\bibitem{chiaveCitazione}
  Andrew S. Tanenbaum,
  \emph{Modern Operating Systems},
  2nd edition,
  Massachusetts,
  Pearson, 
  2nd edition,
  1994.

\end{thebibliography}

\end{document}
